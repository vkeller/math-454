\subsection{Some remarks}

\begin{frame}[containsverbatim]
\frametitle{\textcolor{red}{What this course IS NOT}}

\begin{itemize}
\color{red}

	\item a course about programmation. Dr. Guillaume Anciaux lecture ``Programming concepts in scientific computing'' (or similar level) is a prerequiste.

	\item no ``bleeding-edge'' programmation langages (such as Julia, Rust, Go or alike)

	\item no PGAS langages (such as UPC, CAF, Fortress, HPF or alike)

	\item no high-level parallel programming concepts or tools (such as Cilk, or alike)

	\item no auto parallelization compilers or flags (such as yucca, par4all, ``icc -parallel'' or alike)

\end{itemize}
\end{frame}


\begin{frame}[containsverbatim]
\frametitle{\textcolor{green}{What this course IS}}

\begin{itemize}
\color{green}
	\item theory of parallel computing

	\item beeing able to understand how it work behind the scene

	\item understanding the concepts behind data and instruction parallelisms

	\item How to use shared memory, distributed memory and accelerators paradigms efficiently

	\item understanding how to chose a target architecture based on the needs of an (real) application

	\item you will learn MPI, OpenMP and CUDA under the three main langages used in the world of scientific computing : C, C++ and Fortran.

\end{itemize}

\end{frame}

\begin{frame}[containsverbatim]
\frametitle{How to get the credits}

\begin{itemize}

	\item Follow the courses. Do not hesitate to interrupt the lecturer if you may have a general question that could interest everyone

	\item The exercises are key ! Without getting your hands dirty, you'll not make progress. TA's are there to answer all your questions.

	\item Exercises are exercises. Project is project. You'll have more than 4 times 4 hours to ask questions about the project plus possibilities to send mail to get answers. 

\end{itemize}


\end{frame}


