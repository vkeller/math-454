\documentclass{article}
\usepackage{math}

\title{\sc Poisson's Equation}
\date{}

\begin{document}
\maketitle

Poisson's equation is
\begin{equation}
  \nabla^2 u(r) = v(r),
\end{equation}
where the solution is generally required in some simply-connected volume bounded by a closed surface $S$.  Dirichlet boundary conditions specify $u$ on the boundary of the surface.  Neumann boundary conditions specify the normal gradient $\nabla u \cdot dS$.

For example, in 1-D, suppose
\begin{equation}
  \frac{d^2u(x)}{dx^2} = v(x),
\end{equation}
for $x_l \le x \le x_h$, subject to $u(x_l)=u_l$ and $u(x_h)=u_h$.

To numerically solve the above equation, we first discretize to form grid points
\begin{equation}
  x_i = x_l + \frac{i(x_h-x_l)}{n+1}, i=1,...,n.
\end{equation}
We approximate the differential term using central differences
\begin{equation}
  \frac{d^2u(x_i)}{dx^2} \approx \frac{u_{i-1}-2u_i+u_{i+1}}{(\delta x)^2},
\end{equation}
where $\delta x = (x_h-x_l)/(n+1)$.  We then have
\begin{equation}
  u_{i-1}-2u_i+u_{i+1} = v_i (\delta x)^2, i=1,...,n,
\end{equation}
where $v_i=v(x_i)$.  The boundary conditions are $u_0=u_l$ and $u_{n+1}=u_h$.

We can write the formulation above in matrix form
\begin{equation}
  Au = w,
\end{equation}
where $u=(u_1,u_2,...,u_n)$, $w=(v_1(\delta x)^2-u_l,v_2(\delta x)^2,v_3(\delta x)^2,...,v_{n-1}(\delta x)^2,v_n(\delta x)^2-u_h)$, and for $n=6$,
\begin{equation}
  A =
\begin{pmatrix}
-2&1&0&0&0&0\\
1&-2&1&0&0&0\\
0&1&-2&1&0&0\\
0&0&1&-2&1&0&\\
0&0&0&1&-2&1
\end{pmatrix}.
\end{equation}

The matrix $A$ is tri-diagonal, and can be solved more efficiently $O(n)$ than by direct inversion $O(n^3)$.  Let $a, b, c$ be the below-diagonal, on-diagonal, above-diagonal elements of $A$.  Each row of the linear system is
\begin{equation}
  a_i u_{i-1} + b_i u_i + c_i u_{i+1} = w_i, i=1,...,n,
\end{equation}
where $a_1=c_n=0$.  Assume the solution is of the form
\begin{equation}
  u_{i+1} = x_i u_i + y_i.
\end{equation}
Then
\begin{equation}
  a_i u_{i-1} + b_i u_i + c_i (x_i u_i + y_i) = w_i.
\end{equation}
Solving for $u_i$ gives
\begin{equation}
  u_i = -\frac{a_i u_{i-1}}{b_i + c_i x_i} + \frac{w_i - c_i y_i}{b_i + c_i x_i}.
\end{equation}
Writing the assume solution for $u_i$ gives
\begin{equation}
  u_i = x_{i-1} u_{i-1} + y_{i-1}.
\end{equation}
Then pattern matching the above two equations we have
\begin{align}
  x_{i-1} & = -\frac{a_i}{b_i + c_i x_i} \\
  y_{i-1} & = \frac{w_i - c_i y_i}{b_i + c_i x_i}.
\end{align}
To solve the system, we first compute $x_i$ and $y_i$ for $i=n-1,...,0$, noting that $x_0=0$ and $y_0=(w_1-c_1y_1)/(b_1+c_1x_1)$.  Then we compute $u_i$ for $i=1,2,...,n$.

\bibliographystyle{plain}
\bibliography{references}
\end{document}
