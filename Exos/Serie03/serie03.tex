\documentclass[11pt,a4paper]{article}
\usepackage{a4wide}
\usepackage{enumerate}
\usepackage{enumitem}
\usepackage{pcptex}
\usepackage{xspace}
\usepackage{xcolor}
\usepackage{caption}
\usepackage{listings}
\lstset{basicstyle=\ttfamily,
  showstringspaces=false,
  commentstyle=\color{red},
  keywordstyle=\color{blue}
}

\definecolor{mGreen}{rgb}{0,0.6,0}
\definecolor{mGray}{rgb}{0.5,0.5,0.5}
\definecolor{mPurple}{rgb}{0.58,0,0.82}
\definecolor{backgroundColour}{rgb}{0.95,0.95,0.92}

\lstdefinestyle{CStyle}{
    backgroundcolor=\color{backgroundColour},   
    commentstyle=\color{mGreen},
    keywordstyle=\color{magenta},
    numberstyle=\tiny\color{mGray},
    stringstyle=\color{mPurple},
    basicstyle=\footnotesize,
    breakatwhitespace=false,         
    breaklines=true,                 
    captionpos=b,                    
    keepspaces=true,                 
    numbers=left,                    
    numbersep=5pt,                  
    showspaces=false,                
    showstringspaces=false,
    showtabs=false,                  
    tabsize=2,
    language=C
}

\begin{document}

\enoncetitle{3}

\noindent

\begin{exercise}[Theoretical roofline]
  $~$ % ugly trick to get a newline

Allocate a node and run
\begin{lstlisting}[language=bash]
$> srun -n 1 cat /proc/cpuinfo
\end{lstlisting}

To determine the theoretical peak performance of a core:

\begin{enumerate}
\item compute the theoretical performance of the memory
\item compute the roofline model, in particular the ridge point
\end{enumerate}

\end{exercise}

\begin{exercise}[measured roofline]
  $~$ % ugly trick to get a newline

\begin{itemize}
\item compile and run the code in Stream to compute the sustained memory performance
\item compile and run the code in Dgemm to cimpute the sustained peak performance 
\item compute the roofline model, in particular the ridge point
\end{itemize}

How to run:

\begin{lstlisting}[language=bash,basicstyle=\tiny]
$> module load intel
$> export KMP_AFFINITY=compact,granularity=fine
$> export OMP_NUM_THREADS=7
$> srun -n 1 -N 1 -p serial --reservation=phpc2021 -A phpc2021 --cpus-per-task $OMP_NUM_THREADS ./stream
$> srun -n 1 -N 1 -p serial --reservation=phpc2021 -A phpc2021 --cpus-per-task $OMP_NUM_THREADS ./dgemm
\end{lstlisting}

How do you justify the difference?

\end{exercise}

\begin{exercise}[Jacobi stencil]
  $~$ % ugly trick to get a newline

We want to solve the jacobi stencil:

\begin{lstlisting}[language=C]
u(i,j) = 1/4*(u(i-1, j) + u(i + 1, j) + u(i, j - 1) + u(i, j + 1)
\end{lstlisting}

\begin{enumerate}
\item What is the arithmetic intensity of this equation
\item According to the roofline model, what is the maximum performance we can get
\end{enumerate}

Go to the directory Jacobi:

\begin{itemize}
\item \texttt{jacobi.c} is the main driver
\item \texttt{jacobi-naive.c} is the ``classic'' implementation
\end{itemize}

Compile and run this code using the makefile. 

\begin{enumerate}
\item What is the difference between the reported and the target performance?
\item How can performance be improved?
\end{enumerate}

\end{exercise}


\begin{exercise}[If you finish early...]
  $~$ % ugly trick to get a newline


\texttt{jacobi-sse.c} and \texttt{jacobi-avx*.c} contain different implementation using DLP with intrinsics. 

\begin{enumerate}
\item Which one is the fastest and why? Can you beat those implementations ?
\end{enumerate}

\end{exercise}


\end{document}

